\documentclass[a5paper, portrait, 10pt]{mwart}
%\documentclass[a5paper]{article}
\usepackage[lyric, onesongcolumn]{songs}
\usepackage{polski}
\usepackage[utf8]{inputenc}
\usepackage[bookmarks]{hyperref}

%\usepackage[pdftex]{graphicx}
%\usepackage{rotating}

\setlength{\oddsidemargin}{0in}
\setlength{\evensidemargin}{0in}
%\setlength{\textwidth}{6.5in}
\setlength{\topmargin}{0in}
\setlength{\topskip}{0in}
\setlength{\headheight}{0in}
\setlength{\headsep}{0in}
\setlength{\textheight}{6,5in}
%\settowidth{\versenumwidth}{1.\ }


\newindex{titleidx}{titleidx}
\noversenumbers

\begin{document}%=================================================================================
\author{Mateusz Lewicki}
\title{Koszaliński śpiewnik kolęd}
%
\showindex[1]{Spis pieśni}{titleidx}
\songsection{Kolędy}

\begin{songs}{titleidx}


%=================================================================================
% \beginsong{Tytuł główny \\ tytuł drugi}[by={autor1 and autor2},
%                      sr={Revelation 5:13}, ??
%                      cr={informacje o prawach (copyright info)},
%                      index={alternatywna pozycja w spisie}]
% \beginverse
% \[G]Praise God, \[D]from \[Em]Whom \[Bm]all \[Em]bless\[D]ings \[G]flow;
% \[G]Praise Him, all \[D]crea\[Em]tures \[C]here \[G]be\[D]low;
% \[Em]Praise \[D]Him \[G]a\[D]bove, \[G]ye \[C]heav'n\[D]ly \[Em]host;
% \[G]Praise Fa\[Em]ther, \[D]Son, \[Am]and \[G/B G/C]Ho\[D]ly \[G]Ghost.
% \[C]A\[G]men.
% \endverse
% \endsong
%=================================================================================
\beginsong{Ach ubogi żłobie}
\beginverse
Ach, ubogi żłobie, cóż ja widzę w tobie?
Droższy widok niż ma niebo, w maleńkiej osobie.
\endverse
\beginverse
Zbawicielu drogi, Takżeś to ubogi!
Opuściłeś śliczne niebo, obrałeś barłogi.
\endverse
\beginverse
Czyżeś nie mógł sobie, w największej ozdobie Obrać 
pałacu drogiego, nie w tym leżeć żłobie?
\endverse
\beginverse
Gdy na świat przybywasz, grzechy z niego zmywasz,
A na zmycie tej sprośności gorzkie łzy wylewasz.
\endverse
\beginverse
Któż tu nie struchleje, wszystek nie zdrętwieje?
Któż Cię widząc płaczącego łzą się nie zaleje?
\endverse
\beginverse
Na twarz upadamy, czołem uderzamy,
Witając Cię w tej stajence między bydlętami.
\endverse
\beginverse
Zmiłuj się nad nami, obmyj z grzechów łzami,
Przyjmij serca te skruszone, które Ci składamy.
\endverse
\endsong

%=================================================================================

\beginsong{Anioł pasterzom mówił}[by={słowa z XVI wieku}]
\beginverse
    Anioł pasterzom mówił:
    Chrystus się wam narodził
    W Betlejem, nie bardzo podłym mieście.
    Narodził się w ubóstwie
    Pan waszego stworzenia.
\endverse
\beginverse
    Chcąc się dowiedzieć tego
    Poselstwa wesołego,
    Bieżeli do Betlejem skwapliwie.
    Znaleźli dziecię w żłobie,
    Maryję z Józefem.
\endverse
\beginverse
    Taki Pan chwały wielkiej,
    Uniżył się Wysoki,
    Pałacu kosztownego żadnego
    Nie miał zbudowanego
    Pan wszego stworzenia.
\endverse
\beginverse
    O dziwne narodzenie,
    Nigdy niewysławione!
    Poczęła Panna Syna w czystości,
    Porodziła w całości
    Panieństwa swojego.
\endverse
\beginverse
    Już się ono spełniło,
    Co pod figurą było:
    Arona różdżka ona zielona
    Stała się nam kwitnąca
    I owoc rodząca.
\endverse
\beginverse
    Słuchajcież Boga Ojca,
    Jako wam Go zaleca:
    Ten ci jest Syn najmilszy, jedyny,
    W raju wam obiecany,
    Tego wy słuchajcie.
\endverse
\beginverse
    Bogu bądź cześć i chwała,
    Która byś nie ustała,
    Jako Ojcu, tak i Jego Synowi
    I Swiętemu Duchowi,
    W Trójcy jedynemu.
\endverse
\endsong
%=================================================================================
\beginsong{Bóg się rodzi}[by={Franciszek Karpiński (1792)}]
\beginverse
    Bóg się rodzi, moc truchleje,
    Pan niebiosów obnażony;
    Ogień krzepnie, blask ciemnieje,
    Ma granice Nieskończony.
    Wzgardzony, okryty chwałą;
    Śmiertelny, Król nad wiekami!
\endverse
\beginchorus
	A Słowo Ciałem się stało
	i mieszkało między nami.
\endchorus
\beginverse
    Cóż, niebo, masz nad ziemiany?
    Bóg porzucił szczęście Swoje,
    Wszedł między lud ukochany,
    Dzieląc z nim trudy i znoje.
    Niemało cierpiał, niemało,
    Żeśmy byli winni sami!
\endverse
\beginchorus
	A Słowo...
\endchorus
\beginverse
    W nędznej szopie urodzony,
    Żłób Mu za kolebkę dano!
    Cóż jest? Czym był otoczony?
    Bydło, pasterze i siano!
    Ubodzy, was to spotkało,
    Witać Go przed bogaczami!
\endverse
\beginchorus
	A Słowo...
\endchorus
\beginverse
    Podnieś rękę, Boże Dziecię,
    Błogosław Ojczyznę miłą,
    W dobrych radach, w dobrym bycie
    Wspieraj jej siłę swą siłą.
    Dom nasz i majętność całą,
    I wszystkie wioski z miastami!
\endverse
\beginchorus
	A Słowo...
\endchorus
\beginverse
    Potem i króle widziani
    Cisną się między prostotą,
    Niosąc dary Panu w dani:
    Mirrę, kadzidło i złoto.
    Bóstwo to razem zmieszało
    Z wieśniaczymi ofiarami!
\endverse
\beginchorus
	A Słowo...
\endchorus
\endsong

%=================================================================================
\beginsong{Bracia patrzcie jeno}
\beginverse
    Bracia, patrzcie jeno, jak niebo goreje!
    Znać, że coś dziwnego w Betlejem się dzieje.
    Rzućmy budy, łączmy stada (lub "warty, stada")
    Niechaj nimi Pan Bóg włada, a my do Betlejem, do Betlejem
\endverse
\beginverse
    Patrzcie, jak tam gwiazda światłem swoim miga,
    Pewnie dla uczczenia Pana swego ściga.
    Krokiem śmiałym i wesołym
    Śpieszmy i uderzmy czołem
    Przed Panem w Betlejem, w Betlejem.
\endverse
\beginverse
    Wszakże powiedziałem, że cuda ujrzymy:
    Dziecię, Boga świata, w żłobie zobaczymy.
    Patrzcie, jak biednie okryte
    W żłobie Panię znakomite
    W szopie przy Betlejem, przy Betlejem.
\endverse
\beginverse
    Jak prorok powiedział: Panna zrodzi syna,
    Dla ludu całego szczęśliwa nowina:
    Nam zaś radość w tej tu chwili,
    Gdybyśmy Pana zobaczyli
    W szopie przy Betlejem, przy Betlejem.
\endverse
\beginverse
    Betlejem miasteczko w Juda sławnym będzie,
    Pamiętnym się stanie w tym kraju i wszędzie.
    Ucieszmy się więc ziomkowie,
    Pana tego już uczniowie
    W szopie przy Betlejem, przy Betlejem.
\endverse
\beginverse
    Obchodząc pamiątkę odwiedzin pasterzy,
    Każdy czciciel Boga, co w Chrystusa wierzy,
    Niech się cieszy i raduje,
    Że Zbawcę swego znajduje
    W szopie przy Betlejem, przy Betlejem.
\endverse
\endsong
%=================================================================================
\beginsong{Cicha noc}[by={Joseph Mohr (1816), tłumaczenie Piotr Maszyński}]
\beginverse
    Cicha noc, święta noc!
    Pokój niesie ludziom wszem,
    A u żłobka Matka Święta
    Czuwa sama uśmiechnięta
    Nad Dzieciątka snem,
    Nad Dzieciątka snem.
\endverse
\beginverse
    Cicha noc, święta noc!
    Pastuszkowie od swych trzód
    Biegną wielce zadziwieni
    Za anielskich głosem pieni,
    Gdzie się spełnił cud,
    Gdzie się spełnił cud.
\endverse
\beginverse
    Cicha noc, święta noc!
    Narodzony Boży Syn,
    Pan wielkiego majestatu
    Niesie dziś całemu światu
    Odkupienie win,
    Odkupienie win.
\endverse
\beginverse
    Cicha noc, święta noc,
    Jakiż w tobie dzisiaj cud,
    W Betlejem dziecina święta
    Wznosi w górę swe rączęta
    Błogosławi lud,
    Błogosławi lud.
\endverse
\endsong
%=================================================================================
\beginsong{Dnia jednego o północy}
\beginverse
    Dnia jednego o północy,
    Gdym zasnął w ciężkiej niemocy,
\endverse 
  \beginchorus
      Nie wiem, czy na jawie, czy mi się śniło,
      Że wedle mej budy słońce świeciło.
  \endchorus
\beginverse
    Więc się czym prędzej zerwałem
    I na braci zawołałem;
\endverse 
  \beginchorus
      Na Kubę, na Maćka i na Kaźmierza,
      By wstali czem prędzej mówić pacierza.
  \endchorus
\beginverse
    Nie zaraz się podźwignęli,
    Bo byli twardo zasnęli;
\endverse 
  \beginchorus
      Alem ich po trochu wziął za czuprynę,
      By wstali, bieżeli witać Dziecinę.
  \endchorus
\beginverse
    Porwali się, biegli drogą,
    Gdzie widzieli jasność srogą:
\endverse 
  \beginchorus
      W Betlejem miasteczku, gdzie Dziecię było,
      Które się dla wszystkich na świat zjawiło.
  \endchorus
\beginverse
    Wbiegliśmy zaraz do szopy,
    Uściskaliśmy Mu stopy;
\endverse
  \beginchorus
      Jam dobył fujary, a Kuba rogu,
      Graliśmy wesoło na chwałę Bogu.
  \endchorus
\endsong
%=================================================================================
\beginsong{Do szopy, hej pasterze}
\beginverse
    Do szopy, hej pasterze,
    Do szopy, bo tam cud,
    Syn Boży w żłobie leży,
    By zbawić ludzki ród.
\endverse 
\beginchorus
        Śpiewajcie Aniołowie,
        Pasterze grajcie Mu,
        Kłaniajcie się królowie,
        Nie zbudźcie Go ze snu!
\endchorus
\beginverse
    Pobiegli pastuszkowie
    Ze swymi dary tam,
    Oddali pokłon korny,
    Bo to ich Bóg i Pan.
\endverse
\beginchorus
        Śpiewajcie Aniołowie...
\endchorus
\beginverse
    O Boże niepojęty,
    Któż pojmie miłość Twą?
    Na sianie wśród bydlęty
    Masz tron i służbę swą.
\endverse
\beginchorus
        Śpiewajcie Aniołowie...
\endchorus
\beginverse
    Padnijmy na kolana,
    To Dziecię to nasz Bóg,
    Uczcijmy niebios Pana,
    Miłości złóżmy dług!
\endverse
\beginchorus
        Śpiewajcie Aniołowie...
\endchorus
\beginverse
    Jezuniu mój najsłodszy,
    Oddaję Tobie się,
    O skarbie mój najdroższy,
    Racz wziąć na własność mnie.
\endverse
\beginchorus
        Śpiewajcie Aniołowie...
\endchorus
\beginverse
    Do szopy, hej pasterze,
    Do szopy wszyscy wraz,
    Syn Boży w żłobie leży,
    Więc spieszcie póki czas!
\endverse
\beginchorus
        Śpiewajcie Aniołowie...
\endchorus
\endsong
%=================================================================================
\beginsong{Dzisiaj w Betlejem}[by={XVII w.}]
\beginverse
    Dzisiaj w Betlejem, dzisiaj w Betlejem wesoła nowina,
    Że Panna czysta, że Panna czysta porodziła Syna.
\endverse 
\beginchorus
    Chrystus się rodzi, nas oswobodzi,
    Anieli grają, króle witają,
    Pasterze śpiewają, bydlęta klękają,
    Cuda, cuda ogłaszają.
\endchorus
\beginverse
    Maryja Panna, Maryja Panna Dzieciątko piastuje,
    I Józef stary, i Józef stary Ono pielęgnuje.
\endverse 
\beginchorus
    Chrystus się rodzi, nas oswobodzi...
\endchorus
\beginverse
    Choć w stajeneczce, choć w stajeneczce Panna syna rodzi,
    Przecież On wkrótce, przecież On wkrótce ludzi oswobodzi.
\endverse 
\beginchorus
    Chrystus się rodzi, nas oswobodzi...
\endchorus
\beginverse
    I Trzej Królowie, i Trzej Królowie od wschodu przybyli,
    I dary Panu, i dary Panu kosztowne złożyli.
\endverse 
\beginchorus
    Chrystus się rodzi, nas oswobodzi...
\endchorus
\beginverse
    Pójdźmy też i my, pójdźmy też i my przywitać Jezusa,
    Króla na króle, Króla nad króle uwielbić Chrystusa.
\endverse 
\beginchorus
    Chrystus się rodzi, nas oswobodzi...
\endchorus
\beginverse
    Bądźże pochwalon, bądźże pochwalon dziś, nasz wieczny Panie,
    Któryś złożony, któryś złożony na zielonym sianie.
\endverse 
\beginchorus
    Chrystus się rodzi, nas oswobodzi...
\endchorus
\beginverse
    Bądź pozdrowiony, bądź pozdrowiony Boże nieskończony,
    Wsławimy Ciebie, wsławimy Ciebie, Boże niezmierzony.
\endverse 
\beginchorus
    Chrystus się rodzi, nas oswobodzi...
\endchorus
%\includegraphics[width=0.5 \textwidth]{kolednicy} 
\endsong
%=================================================================================
\beginsong{Gdy się Chrystus rodzi}[by={ok. XIX w.}]
\beginverse
    Gdy się Chrystus rodzi i na świat przychodzi
    Ciemna noc w jasności promienistej brodzi
    Aniołowie się radują, pod niebiosa wyśpiewują:
\endverse
\beginchorus
        Gloria, gloria, gloria in excelsis Deo
\endchorus
\beginverse
    Mówią do pasterzy, którzy trzód swych strzegli,
    Aby do Betlejem czym prędzej pobiegli,
    Bo się narodził Zbawiciel, Wszego Świata odkupiciel
\endverse
\beginchorus
        Gloria, gloria, gloria in excelsis Deo
\endchorus
\beginverse
    O niebieskie Duchy i posłowie nieba,
    Powiedzcież wyraźnie, co nam czynić trzeba
    Bo my nic nie pojmujemy, ledwo od strachu żyjemy
\endverse
\beginchorus
        Gloria, gloria, gloria in excelsis Deo
\endchorus
\beginverse
    Idźcież do Betlejem, gdzie Dziecię zrodzone,
    W pieluszki powite, w żłobie położone
    Oddajcie Mu pokłon Boski, on osłodzi Wasze troski
\endverse
\beginchorus
        Gloria, gloria, gloria in excelsis Deo
\endchorus
\endsong
%=================================================================================
\beginsong{Gdy śliczna Panna Syna kołysała}[by={słowa z XVIII wieku}]
\beginverse
    Gdy śliczna Panna Syna kołysała,
    Z wielkim weselem tak Jemu śpiewała:

        Lili lili laj, moje Dzieciąteczko,
        Lili lili laj, śliczne Paniąteczko.
\endverse
\beginverse
    Wszystko stworzenie, śpiewaj Panu swemu,
    Pomóż radości wielkiej sercu memu.

        Lili lili laj, wielki Królewicu,
        Lili lili laj, niebieski Dziedzicu.
\endverse
\beginverse
    Sypcie się z nieba, śliczni Aniołowie,
    Śpiewajcie Panu, niebiescy duchowie.

        Lili lili laj, mój wonny Kwiateczku,
        Lili lili laj, w ubogim żłóbeczku.
\endverse
\beginverse
    Cicho wietrzyku, cicho południowy,
    Cicho powiewaj, niech śpi Panicz nowy.

        Lili lili laj, mój wdzięczny Synaczku,
        Lili lili laj, miluchny robaczku.
\endverse
\beginverse
    Śpijże już wdzięcznie, moja perło droga,
    Niech Ci snu nie rwie żadna przykra trwoga.

        Lili lili laj, mój śliczny rubinie,
        Lili lili laj, póki sen nie minie.
\endverse
\endsong
%=================================================================================
\beginsong{Hej, w Dzień Narodzenia}[by={ok. XVIII w.}]
\beginverse
    Hej, w Dzień Narodzenia Syna Jedynego
    Ojca Przedwiecznego, Boga prawdziwego:
\endverse		

\beginchorus
        Wesoło śpiewajmy, chwałę Bogu dajmy.
        Hej kolęda! Kolęda!
\endchorus
% % \endverse
\beginverse
    Panna porodziła niebieskie Dzieciątko,
    W żłobie położyła małe Pacholątko.
\endverse		
\beginchorus
        Pasterze śpiewają, na multankach grają.
        Hej kolęda! Kolęda!	
\endchorus	
\beginverse
    Skoro pastuszkowie o tym usłyszeli,
    Zaraz do Betlejem czem prędzej bieżeli.
\endverse
\beginchorus
        Witając Dzieciątko, małe Pacholątko.
        Hej kolęda! Kolęda!
\endchorus
\beginverse
    I tak wszyscy społem wokoło stanęli,
    Panu maleńskiemu wesoło krzyknęli:
\endverse
\beginchorus
	Funda, funda, funda, tota risibunda,
        Hej kolęda! Kolęda!
\endchorus
\beginverse
    Kuba nieboraczek nierychło przybieżał,
    Śpieszno bardzo było, wszystkiego odbieżał.
\endverse
\beginchorus
        Panu nie miał co dać, kazali mu śpiewać.
        Hej kolęda! Kolęda!
\endchorus
\beginverse
    Dobył tak wdzięcznego głos baraniego,
    Że się Józef stary przestraszył od niego,
\endverse
\beginchorus
    Już uciekać myśli, ale drudzy przyszli.
    Hej kolęda! Kolęda!
\endchorus
\beginverse
    Mówi mu Staruszek: nie śpiewaj tak pięknie,
    Bo się głosu twego Dzieciątko przelęknie,
\endverse
\beginchorus
        Lepiej Mu zagrajcie, Panu chwałę dajcie.
        Hej kolęda! Kolęda!
\endchorus
\endsong
%=================================================================================
\beginsong{Jezus malusieńki}
\beginverse
    Jezus malusieńki, leży wśród stajenki,
    Płacze z zimna, nie dała Mu Matula sukienki.
\endverse
\beginverse
    Bo uboga była, rąbek z głowy zdjęła,
    W który Dziecię owinąwszy, siankiem Je okryła.
\endverse
\beginverse
    Nie ma kolebeczki, ani poduszeczki,
    We żłobie Mu położyła, sianka pod główeczki.
\endverse
\beginverse
    Gdy Dziecina kwili, parzy każdej chwili,
    W nóżki zimno, żłóbek twardy, stajenka się chyli.
\endverse
\beginverse
    Matusia truchleje, serdeczne łzy leje:
    O, mój Synu, wola Twoja, nie moja się dzieje.
\endverse
\beginverse
    Przestań płakać proszę, bo żalu nie zniosę,
    Dosyć go mam z męki Twojej, którą w sercu noszę.
\endverse
\beginverse
    Pokłon oddawajmy, Bogiem Je wyznajmy,
    To Dzieciątko ubożuchne ludziom ogłaszajmy.
\endverse
\beginverse
    Niech Je wszyscy znają, serdecznie kochają,
    Za tak wielkie poniżenie chwałę Mu oddają.
\endverse
\beginverse
    O najwyższy Panie! Waleczny Hetmanie!
    Zwyciężonyś, mając rączki miłością związane.
\endverse
\beginverse
    Leżysz na tym sianie, Królu nieba, ziemi,
    Jak baranek na zabicie za moje zbawienie.
\endverse
\beginverse
    Pójdź do serca mego, Tobie otwartego,
    Przysposób je do mieszkania i wczasu Swojego.
\endverse
\beginverse
    Albo mi daj Swoje, wyrzuciwszy moje,
    Tak będziesz miał godny pałac na mieszkanie Twoje. 
\endverse
\endsong
%=================================================================================
\beginsong{Lulajże Jezuniu}
\beginverse
    Lulajże, Jezuniu, moja perełko,
    Lulaj, ulubione me pieścidełko.
\endverse
\beginchorus
        Lulajże, Jezuniu, lulajże, lulaj,
        A Ty Go, Matulu, w płaczu utulaj.
\endchorus
\beginverse
    Zamknijże znużone płaczem powieczki,
    Utulże zemdlone łkaniem usteczki
\endverse
\beginverse
    Dam ja Jezusowi słodkich jagódek,
    Pójdę z nim w Matuli serca ogródek.
\endverse
\beginverse
    Dam ja Jezusowi z chlebem masełka,
    Włożę ja kukiełkę w jego jasełka.
\endverse
\beginverse
    Lulajże, piękniuchny mój aniołeczku,
    Lulajże, maluchny świata kwiateczku.
\endverse
\beginverse
    Matuniu kochana, już odchodzimy,
    Małemu Dzieciątku przyśpiewujemy.
\endverse
\beginverse
    Cyt cyt cyt, już zaśnie małe Dzieciątko,
    Patrz jeno, jak to śpi niby kurczątko.
\endverse
\beginverse
    Cyt cyt cyt, wszyscy się spać zabierajcie,
    Mojego Dzieciątka nie przebudzajcie.
\endverse
\endsong
%=================================================================================
\beginsong{Mędrcy świata, monarchowie}
\beginverse
    Mędrcy świata, monarchowie,
    Gdzie śpiesznie dążycie?
    Powiedzcież nam, Trzej Królowie,
    Chcecie widzieć Dziecię?
\endverse
\beginchorus
    Ono w żłobie nie ma tronu
    I berła nie dzierży,
    A proroctwo Jego zgonu
    Już się w świecie szerzy.
\endchorus
\beginverse
    Mędrcy świata, złość okrutna
    Dziecię prześladuje,
    Wieść okropna, wieść to smutna,
    Herod spisek knuje.
\endverse
\beginchorus
    Nic monarchów nie odstrasza,
    Do Betlejem śpieszą,
    Gwiazda Zbawcę im ogłasza,
    Nadzieją się cieszą.
\endchorus
\beginverse
    Przed Maryją stają społem,
    Niosą Panu dary,
    Przed Jezusem biją czołem,
    Składają ofiary.
\endverse
\beginchorus
    Trzykroć szczęśliwi Królowie,
    Któż wam nie zazdrości?
    Cóż my damy, kto nam powie,
    Pałając z miłości?
\endchorus
\endsong
%=================================================================================


\beginsong{Mizerna, cicha}
\beginverse
Mizerna, cicha, stajenka licha, 
Pełna niebieskiej chwały. 
Oto leżący, przed nami śpiący 
W promieniach Jezus mały. 
\endverse
\beginverse
Nad nim anieli w locie stanęli 
I pochyleni klęczą 
Z włosy złotymi, z skrzydła białymi, 
Pod malowaną tęczą. 
\endverse
\beginverse
I oto mnodzy, ludzie ubodzy 
Radzi oglądać Pana, 
Pełni natchnienia, pełni zdziwienia 
Upadli na kolana.
\endverse
\beginverse
Wielkie zdziwienie:wszelkie stworzenie
Cały świat orzeźiony;
Mądrość Mądrości, Światłość Światłości,
Bóg - człowiek tu wcielony!
\endverse
\beginverse
Oto Maryja, czysta lilija, 
Przy niej staruszek drżący 
Stoją przed nami, przed pastuszkami 
Tacy uśmiechający.
\endverse
\endsong
%=================================================================================


\beginsong{Oj, maluśki, maluśki}[by={XVIII w.}]
\beginverse
    Oj, maluśki, maluśki, maluśki kiejby rękawicka,
    Alboli tez jakoby, jakoby kawałecek smycka.
\endverse
\beginchorus
        (Ref.A): Lili-lili-lili laj, lili laj, lili laj.
        (Ref.B): Śpiewajmy i grajmy Mu, dzieciątku, małemu.
\endchorus
\beginverse
    Cy nie lepiej by Tobie, by Tobie siedzieć było w niebie
    Wsak Twój Tatuś kochany, kochany nie wyganiał Ciebie.
\endverse
\beginverse
    Tam pijałeś coś takie, coś takie słodkie małmazyje,
    Tu się Twoja gębusia, gębusia łez gorzkich napije.
\endverse
\beginverse
    Tam kukiełki jadałeś, jadałeś z carnuską i miodem,
    Tu się jeno zasilać, zasilać musis samym głodem.
\endverse
\beginverse
    Tam wciornaska wygoda, wygoda, a tu bieda wsędzie,
    Ta Ci teraz dokuca, dokuca, ta i potem będzie.
\endverse
\beginverse
    Tam Ty miałeś pościółkę, pościółkę i miętkie piernatki,
    Tu na to Twej nie stanie, nie stanie ubozuchnej Matki.
\endverse
\beginverse
    Tam Ci zawse słuzyły, słuzyły prześlicne janioły,
    A tu lezys sam jeden, sam jeden jako palec goły.
\endverse
\beginverse
    Co się więc takiego, takiego Tobie, Panie stało,
    Zeć się na ten kiepski świat, kiepski świat przychodzić zachciało.
\endverse
\beginverse
    Gdybym ja tam jako Ty, jako Ty tak królował sobie,
    Nie chciałby ja przenigdy, przenigdy w tym spocywać żłobie.
\endverse
\beginverse
    Albo się więc, mój Panie, mój Panie, wróć do swej krainy,
    Alboć pozwól się zanieść, się zanieść do nasej chałpiny.
\endverse
\endsong
%=================================================================================
\beginsong{Pójdźmy wszyscy do stajenki}
\beginverse
    Pójdźmy wszyscy do stajenki,
    Do Jezusa i Panienki,
\endverse
\beginchorus
    Powitajmy Maleńkiego
    I Maryję Matkę Jego. 
\endchorus
\beginverse
    Witaj, Jezu ukochany,
    Od Patriarchów czekany.
\endverse
\beginchorus
    Od Proroków ogłoszony,
    Od narodów upragniony. 
\endchorus
\beginverse
    Witaj, Dzieciąteczko w żłobie.
    Wyznajemy Boga w Tobie.
\endverse
\beginchorus
    Coś się narodził tej nocy.
    Byś nas wyrwał z czarta mocy. 
\endchorus
\beginverse
    O szczęśliwi pastuszkowie,
    Któż radość Waszą wypowie?
\endverse
\beginchorus
    Czego ojcowie żądali,
    Wyście pierwsi oglądali.  
\endchorus
\endsong
%=================================================================================
\beginsong{Północ już była}
\beginverse
Północ już była, gdy się zjawiła
nad bliską doliną jasna łuna,
którą zoczywszy i zobaczywszy.,
krzyknął mocno Wojtek na Szymona:
\endverse
\beginchorus
Szymonie kochany, znak to niesłychany,
że całe niebo goreje!
Na braci zawołaj, niechaj wstawają,
Kuba i Mikołaj niech wypędzają
barany i copy, koźlęta i skopy zamknione.
\endchorus
\beginverse
Na te wołania z smacznego spania
porwał się Stach z Grześkiem i spadł z broga.
Maciek truchleje, od strachu mdleje,
woła: uciekajcie, ach, dla Boga!
\endverse
\beginchorus
Grześko żebro złamał, Stach na nogę chromał,
bo ją w kolanie wywinął.
Oj, oj, oj, dla Boga! Pawełek woła:
uciekajcie prędko, gore stodoła!
Pogorzały szopy i pszeniczne snopy, jam zginął!
\endchorus
\beginverse
Leżąc w stodole, patrząc na pole,
ujrzał Bartosz stary Anioły,
które wdzięcznymi głosami swymi
okrzyknęły ziemskie padoły:
\endverse
\beginchorus
Na niebie niech chwała Bogu będzie trwała,
a ludziom pokój na ziemi!
Pasterze wstawajcie, witajcie Pana,
pokłon Mu oddajcie, wziąwszy barana,
skocznie Mu zagrajcie, głosy zaśpiewajcie zgodnymi!
\endchorus
\endsong
%=================================================================================
\beginsong{Przybierzeli do Betlejem pasterze}
\beginverse
    Przybieżeli do Betlejem pasterze,
    Grając skocznie Dzieciąteczku na lirze.
\endverse
\beginchorus
	Chwała na wysokości,
	Chwała na wysokości,
	A pokój na ziemi.
\endchorus
\beginverse
    Oddawali swe ukłony w pokorze
    Tobie z serca ochotnego, o Boże!
\endverse
\beginverse
    Anioł Pański sam ogłosił te dziwy,
    Których oni nie słyszeli, jak żywi.
\endverse
\beginverse
    Dziwili się napowietrznej muzyce
    I myśleli, co to będzie za Dziecię?
\endverse
\beginverse
    Oto mu się wół i osioł kłaniają,
    Trzej królowie podarunki oddają.
\endverse
\beginverse
    I anieli gromadami pilnują
    Panna czysta wraz z Józefem pilnują.
\endverse
\beginverse
    Poznali Go Mesyjaszem być prawym
    Narodzonym dzisiaj Panem łaskawym
\endverse
\beginverse
    My go także Bogiem, Zbawcą już znamy
    I z całego serca wszyscy kochamy.
\endverse
\endsong
%=================================================================================
\beginsong{Tryumfy Króla Niebieskiego}
\beginverse
    Tryumfy Króla Niebieskiego
    Zstąpiły z nieba wysokiego.
\endverse
\beginchorus
    Pobudziły pasterzów,
    Dobytku swego stróżów
    Śpiewaniem, śpiewaniem, śpiewaniem.
\endchorus
\beginverse
    Chwała bądź Bogu w wysokości,
    A ludziom pokój na niskości.
\endverse
\beginchorus
    Narodził się Zbawiciel,
    Dusz ludzkich Odkupiciel
    Na ziemi, na ziemi, na ziemi.
\endchorus
\beginverse
    Że to Bóg, gdy się dowiedzieli,
    Swej trzody w polu odbieżeli,
\endverse
\beginchorus
    Śpiesząc na przywitanie
    Do Betlejemskiej stajni
    Dzieciątka, Dzieciątka, Dzieciątka.
\endchorus
\beginverse
    Zrodziła Maryja Dziewica
    Wiecznego Boga bez rodzica,
\endverse
\beginchorus
    By nas z piekła wybawił,
    A w niebieskich postawił
    Pałacach, pałacach, pałacach.
\endchorus
\beginverse
    Pasterze w podziwieniu stają,
    Tryumfu przyczynę badają:
\endverse
\beginchorus
    Co się nowego dzieje,
    Że tak światłość jaśnieje,
    Nie widząc, nie widząc, nie widząc. 
\endchorus
\endsong
%=================================================================================
\beginsong{W Dzień Bożego Narodzenia}
\beginverse
    W Dzień Bożego Narodzenia
    Radość wszelkiego stworzenia:
    Ptaszki w górę podlatują,
    Jezusowi wyśpiewują, wyśpiewują.
\endverse
\beginverse
    Słowik zaczyna dyszkantem,
    Szczygieł mu wtóruje altem;
    Szpak tenorem krzyknie czasem,
    A gołąbek gruchnie basem, gruchnie basem.
\endverse
\beginverse
    A mazurek ze swym synem
    Świergąc siedzą za kominem;
    Ach cierp, cierp, cierp miły Panie,
    Póki ten mróz nie ustanie, nie ustanie.
\endverse
\beginverse
    Gdy ptactwo Boga uczciło,
    Co żywo się rozproszyło;
    Ludziom dobry przykład dali,
    Ażeby Go uwielbiali, uwielbiali.
\endverse
\endsong

%=================================================================================
\beginsong{W żłobie leży}[by={Piotr Skarga}]
\beginverse
    W żłobie leży, któż pobieży
    Kolędować Małemu
    Jezusowi Chrystusowi,
    Dziś nam narodzonemu?
    %dziś do nas zesłanemu?
\endverse
\beginchorus
        Pastuszkowie, przybywajcie,
        Jemu wdzięcznie przygrywajcie,
        Jako Panu naszemu.
\endchorus
\beginverse
    My zaś sami z piosneczkami
    Za wami pośpieszymy,
    A tam Tego Maleńkiego
    Niech wszyscy zobaczymy:
\endverse
\beginchorus
        Jak ubogo narodzony,
        Płacze w stajni położony,
        Więc Go dziś ucieszymy.
\endchorus
\beginverse
    Naprzód tedy niechaj wszędy
    Zabrzmi świat w wesołości,
    Że posłany jest nam dany
    Emmanuel w niskości.
\endverse
\beginchorus
        Jego tedy przywitajmy,
        Z Aniołami zaśpiewajmy:
        Chwała na wysokości!
\endchorus
\beginverse
    Witaj Panie, cóż się stanie,
    Że rozkosze niebieskie
    Opuściłeś, a zstąpiłeś
    Na te niskości ziemskie?
\endverse
\beginchorus
        Miłość moja to sprawiła,
        Że człowieka wywyższyła
        Pod nieba Empirejskie.
\endchorus
\endsong

%=================================================================================
\beginsong{Wesołą nowinę, bracia słuchajcie}
\beginverse
    Wesołą nowinę, bracia, słuchajcie
    Niebieską Dziecinę ze mną witajcie.
\endverse
\beginchorus
        Jak miła ta nowina!
        Mów, gdzie jest ta Dziecina?
        Byśmy tam pobieżeli i ujrzeli.
\endchorus
\beginverse
    Panna nam powiła Boskie Dzieciątko,
    Pokłonem uczciła to Niemowlątko.
\endverse
\beginchorus
        Jak miła ta nowina...
\endchorus
\beginverse
    Bogu chwałę głoszą na wysokości,
    Pokój ludziom głoszą duchy światłości
\endverse
\beginchorus
        Jak miła ta nowina...
\endchorus

\endsong
%=================================================================================
\beginsong{Wśród nocnej ciszy}[by={XVIII w.}]
\beginverse
    Wśród nocnej ciszy głos się rozchodzi:
    Wstańcie, pasterze - Bóg się wam rodzi!
    Czem prędzej się wybierajcie,
    Do Betlejem pospieszajcie
    Przywitać Pana.
    Przywitać Pana.
\endverse
\beginverse
    Poszli, znaleźli Dzieciątko w żłobie,
    Z wszystkimi znaki danymi sobie.
    Jako Bogu cześć Mu dali,
    A witając zawołali,
    Z wielkiej radości.
    Z wielkiej radości.
\endverse
\beginverse
    Ach, witaj Zbawco, z dawna żądany!
    Tyle tysięcy lat wyglądany;
    Na Ciebie króle, prorocy
    Czekali, a Tyś tej nocy
    Nam się objawił.
    Nam się objawił.
\endverse
\beginverse
    I my czekamy na Ciebie, Pana,
    A skoro przyjdziesz na głos kapłana,
    Padniemy na twarz przed Tobą,
    Wierząc, żeś jest pod osłoną
    Chleba i wina.
    Chleba i wina.
\endverse
\endsong

%=================================================================================

\beginsong{Zagrzmiała, runęła}

\beginverse
Zagrzmiała, runęła w Betlejem ziemia,
nie było, nie było Józefa doma.
\endverse
\beginverse
Kędyżeś, kędyżeś, Józefie, bywał?
W Betlejem, w Betlejem Dzieciątku śpiewał.
\endverse
\beginverse
Wół, osioł, wół, osioł przed nim klękali,
bo swego, bo swego Stwórcę poznali.
\endverse
\beginverse
Beczący, ryczący Panu śpiewali,
pasterze, pasterze w multanki grali.
\endverse
\beginverse
Zmiłuj się, zmiłuj się, nasz wieczny Panie,
bez Ciebie, bez Ciebie nic się nie stanie.
\endverse
\endsong

%=================================================================================
\beginsong{Z narodzenia Pana}
\beginverse
Z Narodzenia Pana dzień dziś wesoły.
Wyśpiewują chwałę Bogu żywioły.
Radość ludzi wszędzie słynie.
Anioł budzi przy dolinie,
Pasterzów co paśli pod borem woły.
\endverse
\beginverse
Wypada wśród nocy ogień z obłoku.
Dumają pasterze w takim widoku.
Każdy pyta: „Co się dzieje?
Czy nie świta, czy nie dnieje?
Skąd ta łuna bije, tak miła oku!"
\endverse
\beginverse
Skoro pastuszkowie głos usłyszeli.
Zaraz do Betlejem prosto bieżeli.
Tam witali w żłobie Pana,
Poklękali na kolana i oddali dary co z sobą wzięli.
\endverse
\beginverse
Odchodzą z Betlejem pełni wesela,
Że już Bóg wysłuchał próśb Izraela,
Gdy tej nocy to widzieli.
Co prorocy widzieć chcieli,
W ciele ludzkim Boga i Zbawiciela.
\endverse
\beginverse
I my z pastuszkami dziś się radujmy,
Chwałę z Aniołami wraz wyśpiewujmy,
Bo ten Jezus z nieba dany,
Weźmie nas między niebiany,
Tylko Go z całego serca miłujmy!
\endverse
\endsong

%=================================================================================

\beginsong{Kaczka pstra}
\beginverse
    Kaczka pstra, Dziatki ma,
\endverse
\beginchorus
    Siedzi sobie na kamieniu,
    Trzyma dudki na ramieniu,
    Kwa kwa kwa, pięknie gra.
\endchorus
\beginverse
    Skowronek jak dzwonek,
\endverse
\beginchorus
    Gdy do nieba się unosi,
    O kolędę pięknie prosi:
    Ćwir, ćwir, ćwir, tak prosi.
\endchorus
\beginverse
    Gęsiorek, Jędorek
\endverse
\beginchorus
    Na bębenku wybijają,
    Pana wdzięcznie wychwalają:
    Gę gę gę, gęgają.
\endchorus
\beginverse
    Słowiczek muzyczek,
\endverse
\beginchorus
    Gdy się głosem popisuje,
    Wesele światu zwiastuje:
    Ciech ciech ciech, zwiastuje.
\endchorus
\beginverse
    Wróblowie stróżowie,
\endverse
\beginchorus
    Gdy nad szopą świergotają,
    Paniąteczku spać nie dają,
    Dziw dziw dziw, nie dają. 
\endchorus
\beginverse
    Czyżyczek, szczygliczek
\endverse
\beginchorus
    Na gardłeczkach jak skrzypeczkach
    Śpiewają Panu w jasłeczkach;
    Lir lir lir, w jasłeczkach.
\endchorus
\endsong
%=================================================================================

\end{songs}

\end{document}
